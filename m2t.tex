\section{Model-to-text transformation}
\label{sec:m2t}

In the following section, we present the \hilecop{} model-to-text
transformation (HM2T) through its formal specification.  The HM2T
generates a \hvhdl{} design out of an SITPN model. It also generates a
structure that relates the elements of the SITPN model to the elements
of the \hvhdl{} design. This structure is called a SITPN-to-\hvhdl{}
binder.  Its formal definition is as follows:

\begin{definition}[SITPN-to-\hvhdl{} design binder]
  \label{def:sitpn-to-hvhdl-binder}
  Given a $sitpn\in{}SITPN$ and a \hvhdl{} design $d\in{}design$, a
  SITPN-to-\hvhdl{} binder $\gamma\in{}WM(sitpn,d)$ is a tuple\\
  ${<}PMap,TMap,CMap,AFMap{>}$ where:
  \begin{itemize}
  \item $PMap\in{}P\rightarrow{}\{id_p~|~\mathtt{comp}(id_p,\mathtt{place},g,i,o)\in{}d.beh\}$
  \item $TMap\in{}T\rightarrow{}\{id_t~|~\mathtt{comp}(id_t,\mathtt{transition},g,i,o)\in{}d.beh\}$
  \item $CMap\in\mathcal{C}\rightarrow\{id_c~|~(\mathtt{in}, id_c, \mathtt{bool})\in{}d.ports\}$
  \item $AFMap\in\mathcal{A}\cup\mathcal{F}\rightarrow\{id_{af}~|~(\mathtt{out}, id_{af}, \mathtt{bool})\in{}d.ports\}$
  \end{itemize}
\end{definition}

For a given binder $\gamma$ and an element of an SITPN structure
$e\in{}P\cup{}T\cup\mathcal{C}\cup\mathcal{A}\cup\mathcal{F}$, we
write $\gamma(e)$ where $e$ is looked up in the appropriate
function. For instance, for a given $f\in\mathcal{F}$, $\gamma(f)$ is
a shorthand notation for $AFMap(f)$ where $\gamma={<}\dots,AFMap{>}$.

\bigskip

The formal specification of the HM2T is expressed as a relation
between the inputs of the transformation, namely a SITPN model and a
bounding function, and its outputs, namely a \hvhdl{} design and a
SITPN-to-\hvhdl{} binder. The relation is written
$HM2T\subseteq{}SITPN\times(P\rightarrow\mathbb{N})\times{}design\times{}WM(sitpn,d)$.
The bounding function, that is, the second parameter of the
specification relation, associates each place of the SITPN model to a
bound in terms of number of tokens. We assume that these bounds have
been obtained through the formal analysis of the input SITPN model.
The bounds represent the maximum number of tokens that a place will
possibly hold at any point of the execution of the model. Of course,
the existence of such a function implies that all the SITPN models
that are passed as inputs to the HM2T are \textit{bounded}
models\footnote{There exists no place that can accumulate an unlimited
  number of tokens in the course of the execution of the model.}. As a
matter of fact, we can prove that the HM2T is not semantic-preserving
when an unbounded SITPN model is passed as an input. The execution of
such a model will lead to an infinite incrementation of the number of
tokens in a given place. This infinite incrementation, while valid in
the mathematical world, can never be mirrored by a \vhdl{}
implementation of the model where all values must be finite.

Definition of the HM2T relation. Here only selected points of the
definition are presented and ilustrated. The full definition of the
HM2T formal specification can be found at \todo{Add ref. to formal
  spec. ref. (Arxiv or equivalent).}. We have implemented the HM2T
relation in \coq{}, and also have written a \coq{} function that
implements the transformation but is yet to be proved sound and
complete regarding its specification. % However, the
% proof of semantic preservation implies that some properties must be
% drawn regarding the implementation of the HM2T as a program.
However, all the properties that must be proved to state that the HM2T
is semantic preserving contribute to the proof of soundness of our
program w.r.t. its specification.

While expressing the HM2T relation, we refer to sets and functions
that are defined below:

\begin{itemize}
\item
  \texttt{input}$(p)=\{t~\vert~\exists\omega~s.t.~post(t,p)=\omega\}$,
  the set of input transitions of a place $p$.
\item
  \texttt{output}$(p)=\{t~\vert~\exists{}\omega,a~s.t.~pre(p,t)=(\omega,a)\}$,
  the set of output transitions of a place $p$.

\item \texttt{acts}$(p)=\{a~\vert~\mathbb{A}(p,a)=\mathtt{true}\}$,
  the set of actions associated with a place $p$.
\item
  \texttt{input}$(t)=\{p~\vert~\exists\omega,a~s.t.~pre(p,t)=(\omega,a)\}$,
  the set of input places of a transition $t$.
\item
  \texttt{output}$(t)=\{p~\vert~\exists\omega~s.t.~post(t,p)=\omega\}$,
  the set of output places of a transition $t$.
\item
  \texttt{conds}$(t)=\{c~\vert~\mathbb{C}(t,c)=1\lor\mathbb{C}(t,c)=-1\}$,
  the set of conditions associated with a transition $t$.
\item
  \texttt{trs}$(c)=\{t~\vert~\mathbb{C}(t,c)=1\lor\mathbb{C}(t,c)=-1\}$,
  the set of transitions to which a condition $c$ is associated.
\item \texttt{pls}$(a)=\{p~\vert~\mathbb{A}(p,a)=\mathtt{true}\}$, the
  set of places to which an action $a$ is associated.
\item \texttt{trs}$(f)=\{t~\vert~\mathbb{F}(t,f)=\mathtt{true}\}$, the
  set of transitions to which a function $f$ is associated.
\end{itemize}

\begin{itemize}
\item \texttt{output}$_c\in{}P\rightarrow{}2^T$.  The
  \texttt{output}$_c$ function takes a place $p$ as input and yields
  an ordered set of transitions computed as follows:
  \begin{enumerate}
  \item If all conflicts between the output transitions of $p$ are
    solved by mutual exclusion, or if the set of conflicting
    transitions of $p$ is a singleton, then \texttt{output}$_c$
    returns an empty set.
  \item Otherwise, the function tries to establish a total ordering
    over the set of conflicting transitions of $p$ w.r.t the firing
    priority relation:
    \begin{itemize}
    \item If no such ordering can be established (in that case, the
      firing priority relation is ill-formed, and the input SITPN is
      not well-defined), \texttt{output}$_c$ raises an error.
    \item Otherwise, the function returns the ordered set, with the
      top-level priority transition at the head.
    \end{itemize}
  \end{enumerate}

\item \texttt{output}$_{nc}\in{}P\rightarrow{}2^T$.  The
  \texttt{output}$_{nc}$ function takes a place $p$ as input and
  yields an unordered set of transitions computed as follows:
  \begin{itemize}
  \item If all conflicts between the output transitions of $p$ are
    solved by mutual exclusion, or if the set of conflicting
    transitions of $p$ is a singleton, then, the function returns the
    set of output transitions of $p$, i.e. \texttt{output}$(p)$ as
    defined above.
  
  \item Otherwise, the function returns the set of output transitions
    of $p$ connected through a \texttt{test} or an \texttt{inhib} arc,
    i.e.
    $\{t~\vert~\exists\omega~s.t.~pre(p,t)=(\omega,\mathtt{test})\lor{}pre(p,t)=(\omega,\mathtt{inhib})\}$.
  \end{itemize}
\end{itemize}


%%%%% Definition for the HM2T formal specification. %%%%%

\def\pdiInBeh{\mathtt{comp}(\gamma(p),\mathtt{place},g_p,i_p,o_p)\in{}d.beh}
\def\tdiInBeh{\mathtt{comp}(\gamma(t),\mathtt{transition},g_t,i_t,o_t)\in{}d.beh}

\begin{definition}[\hilecop{} model-to-text transformation specification]
  \label{def:hm2t-spec}
  For all SITPN model $sitpn\in{}SITPN$, bounding function
  $b\in{}P\rightarrow\mathbb{N}$, \hvhdl{} design $d\in{}design$, and
  SITPN-to-\hvhdl{} binder $\gamma\in{}WM(sitpn,d)$, we have
  $HM2T(sitpn,b,d,\gamma)$ if:
  
  \begin{enumerate}
  \item Design $d$ has an empty generic clause: $d.gens=\emptyset$.
    
  \item Design $d$ is elaborable in the context of the \hilecop{} design store and given an empty dimensioning function:\\
    $\exists{}\Delta\in{}ElDesign,\sigma_e\in\Sigma$ s.t.
    $\mathcal{D_\mathcal{H}},\emptyset\vdash{}d\xrightarrow{elab}\Delta,\sigma_e$.
    
  \item All the fields of the SITPN-to-\hvhdl{} binder are bijective
    functions: $PMap(\gamma)$ is bijective, $TMap(\gamma)$ is
    bijective,\dots
    
  \item For all place of the input SITPN model, there exists a
    corresponding place design instance (PDI) in the behavior of the
    \hvhdl{} design such that the identifier of the PDI is the one
    yielded
    by the binder $\gamma$:\\
    $\forall{}p\in{}P,\exists{}g_p,i_p,o_p$ s.t.  $\pdiInBeh$.
    
  \item For all place of the input SITPN model and its corresponding PDI, the generic map of the PDI holds the following associations:\\
    $\forall{}p\in{}P,g_p,i_p,o_p,\pdiInBeh\Rightarrow$\\
    $g_p=\{(\mathtt{mm}\Rightarrow{}b(p)), (\mathtt{ian}\Rightarrow
    \begin{cases}
      1~\mathrm{if}~\mathtt{input}(p)=\emptyset \\
      \vert{}\mathtt{input}(p)\vert~\mathrm{otherwise} \\
    \end{cases}),
    (\mathtt{oan}\Rightarrow
    \begin{cases}
      1~\mathrm{if}~\mathtt{output}(p)=\emptyset \\
      \vert{}\mathtt{output}(p)\vert~\mathrm{otherwise} \\
    \end{cases})\}$.
    
  \item For all place of the input SITPN model and its corresponding PDI, there is an association between the \texttt{im} input port and the initial marking of the place in the input port map of the PDI:\\
    $\forall{}p\in{}P,g_p,i_p,o_p,\pdiInBeh{}$
    $\Rightarrow(\mathtt{im}\Rightarrow{}M_0(p))\in{}i_p$.
    
  \item For all place of the SITPN model with no input transition, the
    input port map of the corresponding PDI includes the following associations:
    \begin{equation*}
      \begin{aligned}
        \forall{}p&\in{}P,g_p,i_p,o_p, \\
                  & \pdiInBeh\Rightarrow \\
                  & \mathtt{input}(p)=\emptyset\Rightarrow \\
                  & \{(\mathtt{iaw}(0)\Rightarrow{}0), (\mathtt{itf}(0)\Rightarrow{}\mathtt{false})\}\subseteq{}i_p.\\
      \end{aligned}
    \end{equation*}

  \item For all place of the SITPN model with no output transition,
    the input port map and output port map of the corresponding PDI
    includes the following associations:
    \begin{equation*}
      \begin{aligned}
        \forall{}p&\in{}P,g_p,i_p,o_p, \\
                  & \pdiInBeh\Rightarrow \\
                  & \mathtt{output}(p)=\emptyset\Rightarrow \\
                  & \{(\mathtt{oaw}(0)\Rightarrow{}0), (\mathtt{oat}(0)\Rightarrow{}\mathtt{basic}),
                    (\mathtt{otf}(0)\Rightarrow{}\mathtt{false})\}\subseteq{}i_p\\
                  & \land\{(\mathtt{oav}\Rightarrow{}\mathtt{open}), (\mathtt{pauths}\Rightarrow{}\mathtt{open}),
                    (\mathtt{rtt}\Rightarrow{}\mathtt{open})\}\subseteq{}o_p \\
      \end{aligned}
    \end{equation*}

    
  \item For all place of the SITPN model with no action, the
    \texttt{marked} output port is left unconnected in the output port
    map of the corresponding PDI:
    \begin{equation*}
      \begin{aligned}
        \forall{}p&\in{}P,g_p,i_p,o_p, \\
                  & \pdiInBeh\Rightarrow \\
                  & \mathtt{acts}(p)=\emptyset\Rightarrow \\
                  & (\mathtt{marked}\Rightarrow{}\mathtt{open})\in{}o_p \\
      \end{aligned}
    \end{equation*}

  \item For all transition of the input SITPN model, there exists a
    corresponding transition design instance (TDI) in the behavior of
    the \hvhdl{} design such that the identifier of the TDI is the one
    yielded
    by the binder $\gamma$:\\
    $\forall{}t\in{}T,\exists{}g_t,i_t,o_t$ s.t. $\tdiInBeh$.
    
  \item For all transition of the input SITPN model and its corresponding TDI, the generic map of the TDI holds the following associations:\\
    \begin{equation*}
      \begin{aligned}[t]
        \forall{}t&\in{}T,g_t,i_t,o_t, \\
                  & \tdiInBeh\Rightarrow \\
                  & g_t=\{(\mathtt{tt}\Rightarrow{}
                    \begin{cases}
                      \mathtt{NOT\_TEMPORAL}~\mathrm{if}~t\notin{}\mathtt{dom}(I_s) \\
                      \mathtt{TEMPORAL\_A\_A}~\mathrm{if}~I_s(t)=[a,a] \\
                      \mathtt{TEMPORAL\_A\_B}~\mathrm{if}~I_s(t)=[a,b] \\
                      \mathtt{TEMPORAL\_A\_INF}~\mathrm{if}~I_s(t)=[a,\infty] \\
                    \end{cases}),
        (\mathtt{mtc}\Rightarrow
        \begin{cases}
          1~\mathrm{if}~t\notin{}\mathtt{dom}(I_s) \\
          b~\mathrm{if}~I_s(t)=[a,b] \\
          a~\mathrm{if}~I_s(t)=[a,\infty] \\
        \end{cases}), \\
                  & (\mathtt{ian}\Rightarrow
                    \begin{cases}
                      1~\mathrm{if}~\mathtt{input}(t)=\emptyset \\
                      \vert{}\mathtt{input}(t)\vert~\mathrm{otherwise} \\
                    \end{cases}), 
        (\mathtt{cn}\Rightarrow
        \begin{cases}
          1~\mathrm{if}~\mathtt{conds}(t)=\emptyset \\
          \vert{}\mathtt{conds}(t)\vert~\mathrm{otherwise} \\
        \end{cases})\}. \\
      \end{aligned}
    \end{equation*}
    
  \item For all transition of the input SITPN model and its corresponding TDI, the input port map of the TDI holds the following associations:\\
    \begin{equation*}
      \begin{aligned}[t]
        \forall{}t&\in{}T,g_t,i_t,o_t, \\
                  & \tdiInBeh\Rightarrow \\
                  & \{(\mathtt{A}\Rightarrow\begin{cases}
                                              0~\mathrm{if}~t\notin\mathtt{dom}(I_s) \\
                                              l(I_s(t))~\mathrm{otherwise} \\
                                            \end{cases}),
        (\mathtt{B}\Rightarrow\begin{cases}
                                0~\mathrm{if}~t\notin\mathtt{dom}(I_s)\lor{}u(I_s(t))=\infty \\
                                u(I_s(t))~\mathrm{otherwise} \\
                              \end{cases})\}\subseteq{}i_t. \\
      \end{aligned}
    \end{equation*}
    
  \item For all transition of the input SITPN model with no input place, the input port map and output port map of the corresponding TDI holds the following associations:\\
    \begin{equation*}
      \begin{aligned}[t]
        \forall{}t&\in{}T,g_t,i_t,o_t, \\
                  & \tdiInBeh\Rightarrow \\
                  & \mathtt{input}(t)=\emptyset\Rightarrow \\
                  & (\exists{}id_s~s.t.~(id_s,\mathtt{bool})\in{}d.sigs\land(\mathtt{rt}(0)\Rightarrow{}id_s)\in{}i_t
                    \land(\mathtt{fired}\Rightarrow{}id_s)\in{}o_t)\\
                  & \land\{(\mathtt{iav}(0)\Rightarrow\mathtt{true}),
                    (\mathtt{pauths}(0)\Rightarrow\mathtt{true})\}\subseteq{}i_t. \\
      \end{aligned}
    \end{equation*}
    
  \item For all transition of the input SITPN model with no condition, the input port map of the corresponding TDI holds the following association:\\
    \begin{equation*}
      \begin{aligned}[t]
        \forall{}t&\in{}T,g_t,i_t,o_t, \\
                  & \tdiInBeh\Rightarrow \\
                  & \mathtt{conds}(t)=\emptyset\Rightarrow \\
                  & (\mathtt{ic}(0)\Rightarrow\mathtt{true})\in{}i_t. \\
      \end{aligned}
    \end{equation*}
  \end{enumerate}

\end{definition}


%%% Local Variables:
%%% mode: latex
%%% TeX-master: "main"
%%% End:
