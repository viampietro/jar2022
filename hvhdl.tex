\section{A target language: \hvhdl{}}
\label{sec:hvhdl}

The \hilecop{} model-to-text transformation generates a \vhdl{} design
out of an input SITPN model. Therefore, to conduct to proof of
semantic preservation, the syntax and semantics of the \vhdl{}
language must be formally set. The designs generated by the \hilecop{}
transformation rely on subset of the \vhdl{} that we identify as
\hvhdl{} and present in the following abstract syntax.

\subsection{The abstract syntax of \hvhdl{}}
\label{subsec:abs-syntax}

The following subset of \vhdl{} has been determined based on
\hilecop{}'s transformation, and also on the definition of two
pre-defined designs, i.e. the place and transition designs. As it will
be presented in Section~\ref{sec:m2t}, the \hilecop{} transformation
is mainly about instantiating, i.e. creating an instance of a design
acting as a subcomponent in the behavior of an embedding design, the
place and transition designs. Thus, being able to reason about the
\hilecop{}'s generated \vhdl{} designs is being able to reason about
the code of the place and transition designs. This is why the abstract
syntax of \hvhdl{} is closely tied to the content of the place and
transition designs.  We present the syntax of \hvhdl{} in a bottom-up
manner, starting from expressions to higher constructs.

\begin{table}[!htbp]
  \caption{Expressions}
  \label{tab:expr}
  \begin{tabular}{|rll|}
    \hline
    & & \\
    $e$ & ::= $name$ & read a signal, a local variable \\
    & & or a generic constant value \\
    & \quad $\vert{}~cst$ & constant \\
    & \quad $\vert{}~bop$($e_1$, $e_2$) & binary operation \\
    & \quad $\vert{}~uop$($e$) & unary operation \\
    & \quad $\vert{}~$\texttt{(}$e^{+}$\texttt{)} & aggregate expression \\
    & & \\
    $name$ & ::= $id$ & read a signal, local variable, \\
    & & or generic constant value \\
    & \quad$\vert{}~$ $id$\texttt{(}$e$\texttt{)} & read value of an array signal  \\
    & & or local variable at index $e$ \\
    & & \\
    $cst$ & ::= $n$ $\vert{}~$ $b$ & natural or Boolean \\
    & & \\
    $bop$ & ::= \texttt{and} $\vert{}$ \texttt{or} & Boolean operators \\
    & \quad$\vert{}$ \texttt{add} $\vert{}$ \texttt{sub} & natural number arithmetic \\
    & \quad$\vert{}$ \texttt{eq} $\vert{}$ \texttt{ne} $\vert{}$ \texttt{gt} $\vert{}$ \texttt{ge} $\vert{}$ \texttt{lt} $\vert{}$ \texttt{le} & comparisons \\
    & & \\
    $uop$ & ::= \texttt{not} & Boolean negation \\
    & & \\
    \hline
  \end{tabular}
\end{table}

The expressions of \hvhdl{} are restrained to operations over Boolean
or natural numbers that are used in the designs generated by the
transformation, and more particularly the ones used in the pre-defined
place and transition designs.

\begin{table}[!htbp]
  \caption{Sequential statements}
  \label{tab:ss}
  \begin{tabular}{|rll|}
    \hline
    & & \\
    $ss$ & ::= $name~\mathtt{\Leftarrow}~e$ & assignment to a signal \\
    & \quad$\vert{}~name~\mathtt{:=}~e$ & assignment to a local variable \\
    & \quad$\vert{}~\mathtt{if}(e)\{ss_1\}~\mathtt{else}~\{ss_2\}$ & conditional \\
    & \quad$\vert{}~\mathtt{for}(id,e_1,e_2)\{ss\}$ & range loop \\
    & \quad$\vert{}~\mathtt{falling}\{ss\}$ & falling edge block \\
    & \quad$\vert{}~\mathtt{rising}\{ss\}$ & rising edge block \\
    & \quad$\vert{}~\mathtt{rst}~\{ss_1\}~\mathtt{else}~\{ss_2\}$ & reset conditional \\
    & \quad$\vert{}~ss_1\mathtt{;}ss_2$ & sequence \\
    & \quad$\vert{}~\mathtt{null}$ & no operation \\
    & & \\
    \hline
  \end{tabular}
\end{table}

Sequential statements are used to defined the body of processes, and
mainly act upon the value of signals and local variables through
assignment operations. The set of sequential statement includes the
classical conditional, range loop, and sequence statements. We add
particular statements, namely the falling edge, rising edge and reset
conditional statements, derived from the concrete syntax of \vhdl{}.
These statements are convenient to express block of statements to be
executed only at certain phases of the simulation.


\begin{table}[!htbp]
  \caption{Type indication}
  \label{tab:typeind}
  \begin{tabular}{|rll|}
    \hline
    & & \\
    $\tau$ & ::= \texttt{bool} & boolean \\
    & \quad$\vert{}~$ \texttt{nat} \texttt{(}$e_1$\texttt{,} $e_2$\texttt{)} & natural range $e_1$ to $e_2$ \\
    & \quad$\vert{}~$ \texttt{array} \texttt{(}$\tau$\texttt{,} $e_1$\texttt{,} $e_2$\texttt{)} & array of $\tau$ with index range $e_1$ to $e_2$ \\
    & & \\
    \hline
  \end{tabular}
\end{table}

A type indication informs us about the type of a given signal, local
variable, or generic constant at the time of its declaration. A
signal, a local variable or a generic constant can be a Boolean, a
natural number defined in a certain range, or an array of elements
associated with a certain type indication.

\begin{table}[!htbp]
  \caption{Concurrent statements}
  \label{tab:cs}
  \begin{tabular}{|rll|}
    \hline
    & & \\
    $cs$ & ::= $ps$ & process statement \\
    & $\vert{}~$ $comp$ & design instantiation statement \\
    & $\vert{}~$ $cs_1~\mathtt{||}~cs_2$ & parallel composition \\
    & $\vert{}~$ \texttt{null} & no operation \\
    & & \\
    $ps$ & ::= $\mathtt{ps}(id_p,$ & process identifier \\
    & \quad\quad\quad${}vars=\{(id,\tau)^{*}\},$ & local variable declarations\\
    & \quad\quad\quad${}body=ss)$ & statement body \\
    & & \\
    $comp$ & ::= $\mathtt{comp}(id_c,$ & component instance identifier \\
      & \quad\quad\quad\quad$id_e,$ & instantiated design identifier \\
      & \quad\quad\quad\quad${}g=\{(id\Rightarrow{}e)^{*}\},$ & generic constant map \\
      & \quad\quad\quad\quad${}i=\{(name\Rightarrow{}e)^{*}\},$ & input port map \\
    & \quad\quad\quad\quad$o=\{\big((id\Rightarrow{}(name\vert{}\mathtt{open}))$ & output port map \\
    & \quad\quad\quad\quad\quad\quad\quad$\vert{}(id(e)\Rightarrow{}name)\big)^{*}\})$ &  \\
    & & \\
    \hline
  \end{tabular}

\end{table}

The behavior of a design is defined by concurrent statements. A
concurrent statement can be a process, a design instantiation, or the
parallel composition of two concurrent statements. A process statement
declares a set of local variables, and executes operations over
signals and variables defined in its sequential statement body. A
design instantiation statement represents the creation of a
subcomponent having a part in the definition of the embedding
design. A design instantiation statement indicates which design is
instantiated as a subcomponent (i.e. $id_e$). It also indicates how
the subcomponent is dimensioned through a generic constant map
(i.e. $g$). Moreover, it indicates how the subcomponent is connected
to the other parts of the embedding design through a input port
(i.e. $i$) and output port map (i.e. $o$).

\begin{table}[!htbp]
  \caption{Design}
  \label{tab:design}
  \begin{tabular}{|rll|}
    \hline
    & & \\
    $design$ & ::= $\{{}gens=\{(id,\tau,e)^{*}\},$ & generic constants \\
    & \quad\quad${}ports=\{((\mathtt{in}\vert\mathtt{out}),id,\tau)^{*}\},$ & input and output ports \\
    & \quad\quad${}sigs=\{(id,\tau)^{*}\},$ & internal signals \\
    & \quad\quad${}behavior=cs\}$ & design behavior \\
    & & \\
    \hline
  \end{tabular}
\end{table}

The highest construct of the \hvhdl{} language is the design. A design
represents a whole circuit by itself. Once defined, a design can be
later be instantiated to define the behavior of other designs. A
design that is not instantiated as a part of the behavior of another
design is referred to as a \textit{top-level} design.

\todo[inline]{Give an example of \vhdl{} design in concrete and
  abstract syntax.}

\subsection{Simulation semantics}
\label{subsec:sim-semantics}

The \vhdl{} Language Reference Manual\cite{VHDL2000} defines the
semantics of a \vhdl{} design with a simulation algorithm. The
simulation algorithm is informally presented in the LRM. Many
formalization of the algorithm exist in the scientific
literature\cite{Borger1995,Borrione1995,Breuer1994,Breuer1995,Breuer1995a,Deharbe1995,Dohmen1995,Fuchs1995,Goossens1995,Kloos2012,Olcoz1995,Pandey1999,Reetz1995,Shankar1997,Thirunarayan2001,VanTassel1995}. Considering
our specific needs regarding the \vhdl{} designs generated by the
\hilecop{} transformation, a particular simulation semantics has been
defined for the \hvhdl{} language. We give a simpler formal definition
of the simulation algorithm. The simplification of the algorithm is a
consequence of two characteristics of the \hvhdl{} designs. First, the
\hilecop{}-generated \hvhdl{} designs are \textit{synthezisable},
i.e. they target a physical implementation. Thus, certain constructs
of the \vhdl{} language such as wait statements, unit-delay signal
assignments (a.k.a. timed constructs) are not considered, as they can
not lead to a physical implementation. Removing the timed constructs
from the subset of \vhdl{} we are dealing with considerably simplifies
the expression of the simulation algorithm. Second, the
\hilecop{}-generated \hvhdl{} designs are \textit{synchronous},
i.e. the execution of certain parts of their behavior is triggered by
certain events of a clock signal. Thus, the expression of our
simulation algorithm explicitly takes into account the clock signal
events. In what follows, we give a formal definition of the simulation
algorithm that gives a semantics to the execution of \hvhdl{}
designs. Our semantics takes the form of a big-step operational
semantics. Our main inspirations were the operational semantics of
J. Van Tassel\cite{VanTassel1995}, and the denational semantics for a
synchronous subset of \vhdl{} by D. Borrione\cite{Borrione1995}. The
rules of the semantics will be presented in a bottom-up manner,
starting from the evaluation of expressions to the definition of the
whole simulation algorithm.

\subsubsection{Semantic domains}
\label{subsubsec:sem-domains}

Before detailling the semantics rules, we must introduce the $type$
and $value$ sets representing the types and values of the
semantics. We must also introduce the definitions of an elaborated
design, and the notion of design state, as the simulation of a design
computes the evolution of a design state through multiple clock
cycles. We also introduce here some mandatory definitions such as the
typing relation $\in_c$, or the predicate qualifying static
expressions.

\begin{table}[!htbp]
  \caption{The $t$ (type) and $v$ (value) semantic types.}
  \label{tab:type-value}

  \begin{tabular}{|rll|}
    \hline
    && \\
    $t$ & ::= $\mathtt{bool}$ & Boolean type \\
    & \quad $\vert~\mathtt{nat}(n_1$, $n_2)$ & natural range $n_1$ to $n_2$ \\
    & \quad $\vert~\mathtt{array}(t, n_1, n_2)$ & array of $t$ with index range $n_1$ to $n_2$ \\
    & & \\
    $v$ & ::= $b$ & Boolean \\
    & \quad $\vert~{}n$ & natural number (limited to $\mathtt{NATMAX}$) \\
    & \quad $\vert~(v^{+})$ & array of values \\
    && \\
    \hline
  \end{tabular}    
\end{table}

In Table~\ref{tab:type-value}, the $type$ set directly reflects the
type indication set used in the abstract syntax of \hvhdl{}. In the
$type$ set, the bounds of a natural number range and the index range
of an array have all been evaluated to natural numbers. In the
\hvhdl{} semantics, a value can be a Boolean, a natural number, or an
array of values. $\mathtt{NATMAX}$ denotes the maximum value for a
natural number.  The $\mathtt{NATMAX}$ value depends on the
implementation of the \textsf{VHDL} language; $\mathtt{NATMAX}$ must
at least be equal to $2^{31}-1$.

Now, let us define the structure of \textit{elaborated design}. An
elaborated design is built during the elaboration of a \hvhdl{}
design. Then, the elaborated design will act as a runtime environment
in the expression of the simulation rules. The elaboration phase
corresponds to the creation of the global environment of the
simulation. During the elaboration phase as described in the LRM, a
design is statically type-checked, and certain tests are performed
regarding the well-formedness of the design (e.g. connections in port
maps, multiply-driven signals, etc.). In the LRM, the elaboration
phase also involves the transformation of design instantiation
statements into \textit{block} statements; it corresponds to the
flattening of the structure of subcomponents brought by the design
instantiation statements. We have formally defined the elaboration
phase through an elaboration relation. In our formalization, we do not
take into account the transformation of design instantiation
statements into block statements. For the purpose of the proof of
semantic preservation, we are interested in preserving the
hierarchical structure provided by the design instantiation
statements, arguing that it permits us to better relate the structure
of a SITPN model to the structure of the corresponding \hvhdl{} design
obtained through the \hilecop{} transformation. The full formalization
of the elaboration relation is presented here\cite{Iampietro2021}.

Let $ElDesign$ be the set of elaborated designs. An elaborated design
is a composite environment built out of multiple sub-environments.
Each sub-environment is a table, represented as a function, mapping
identifiers of a certain category of constructs (e.g, input port
identifiers) to their declaration information (e.g, type indication
for input ports). We represent an elaborated design as a record where
the fields are the sub-environments. An elaborated design is defined
as follows:

\begin{definition}[Elaborated design]
  \label{def:elab-design}
  An elaborated design $\Delta\in{}ElDesign$ is a record\\
  ${<}Gens, Ins, Outs, Sigs, Ps, Comps{>}$ where:
  \begin{itemize}[label=$-$]
  \item $Gens\in{}generic\mhyphen{}id\rightarrow{}(t\times{}v)$
    is the function yielding the type and the value of generic
    constants.
  \item $Ins\in{}input\mhyphen{}id\rightarrow{}t$ is the function
    yielding the type of input ports.
  \item $Outs\in{}output\mhyphen{}id\rightarrow{}t$ is the function
    yielding the type of output ports.
  \item
    $Sigs\in{}declared\mhyphen{}signal\mhyphen{}id\rightarrow{}t$
    is the function yielding the type of declared signals.
  \item
    $Ps\in{}process\mhyphen{}id\rightarrow(variable\mhyphen{}id\rightarrow{}(t\times{}v))$
    is the function associating process identifiers to their local
    environment.
  \item $Comps\in{}component\mhyphen{}id{}\rightarrow{}ElDesign$ is
    the function mapping component instance identifiers to their own
    elaborated design version.
  \end{itemize}
\end{definition}

We assume that there is no overlapping between the identifiers of the
sub-environments of an elaborate design (i.e, an identifier belongs to
at most one sub-environment), and also between the identifiers of the
sub-environments and the identifiers of local environments. When there
is no ambiguity, we write $\Delta(x)$ to denote the value returned for
identifier $x$, where $x$ is looked up in the appropriate field of
$\Delta$. We write $x\in\Delta$ to state that identifier $x$ is
defined in the domain of one of $\Delta$'s field. We note
$\Delta(x)\leftarrow{}v$ the overriding of the value associated to
identifier $x$ with value $v$ in the appropriate field of $\Delta$,
$\Delta\cup{}(x,v)$ to note the addition of the mapping from
identifier $x$ to value $v$ in the appropriate field of $\Delta$, that
assuming $x\notin\Delta$. We write $x\in\mathcal{F}(\Delta)$, where
$\mathcal{F}$ is a field of $\Delta$, when more precision is needed
regarding the lookup of identifier $x$ in the record $\Delta$.\\

Now let us define the run-time state of a design, i.e. the state that
describes the value of signals and component instances in the course
of a simulation. Let $\Sigma$ be the set of design states.  A design
state of $\sigma\in{}\Sigma$ is defined as follows:

\begin{definition}[Design state]
  \label{def:design-state}
  A design state $\sigma\in\Sigma$ is a record
  ${<}\mathcal{S},\mathcal{C},\mathcal{E}{>}$ where:
  \begin{itemize}[label=$-$]
  \item $\mathcal{S}\in{}signal\mhyphen{}id\rightarrow{}v$ is the
    signal store, i.e. the function yielding the current values of
    ports and declared signals.
  \item $\mathcal{C}\in{}component\mhyphen{}id\rightarrow{}\Sigma$ is
    the component store, i.e.  the function yielding the current state
    of component instances.
  \item
    $\mathcal{E}\subseteq{}signal\mhyphen{}id\cup{}component\mhyphen{}id$,
    is the set of signal and component instance identifiers that
    generated an event at the current design state.
  \end{itemize}
\end{definition}

The $signal\mhyphen{}id$ subset is the union between the
$input\mhyphen{}id$, $output\mhyphen{}id$ and
$declared\mhyphen{}signal\mhyphen{}id$ subsets. When there is no
ambiguity regarding which store a given identifier belongs, we use
$\sigma(id)$ to denote the value associated to an identifier in the
signal store $\mathcal{S}$ or in the component store $\mathcal{C}$
fields.  We write $id\in\sigma$ to state that an identifier is defined
in either the signal store $\mathcal{S}$ or the component store
$\mathcal{C}$ fields. Also, when there is no ambiguity, we rely on
indices or exponents to qualify the signal store, the component
instance store and the set of events of a given design state. For
instance, $\mathcal{C}_0$ denotes the component instance store of
design state $\sigma_0$, and $\mathcal{E}'$ denotes the set of events
of design state $\sigma'$, etc.

\subsubsection{Evaluation of expressions}
\label{subsubsec:expr-eval}



%%% Local Variables:
%%% mode: latex
%%% TeX-master: "main"
%%% End:
