\documentclass[pdflatex,sn-mathphys]{sn-jnl}% Basic Springer Nature Reference Style/Chemistry Reference Style

%%%% Standard Packages and macros (don't forget to paste the macros in
%%%% the main at the end of the writing).
\usepackage{todonotes}
\usepackage{lstvhdl, lstautogobble, lstpseudocoq, lstcoq}
\usepackage{enumitem}
\usepackage{ebproof}
\ebproofset{left label template=\fontsize{9}{11}\selectfont\inserttext}
\ebproofset{right label template=\fontsize{7}{10}\selectfont\inserttext}

%%%%%%%%%%%%%%%%%%%%%%%%%%%%%%%%%%%%%%
%%%%%%% Misc. macros and defs. %%%%%%%
%%%%%%%%%%%%%%%%%%%%%%%%%%%%%%%%%%%%%%

% Macro definitions.

\def\bbook{\textsf{B}-Book}
\def\bmth{\textsf{B}}
\def\cakeml{\textsf{CakeML}}
\def\cnrs{\textsf{CNRS}}
\def\coq{\textsf{Coq}}
\def\java{\textsf{Java}}
\def\hilecop{\textsf{HILECOP}}
\def\inria{\textsf{Inria}}
\def\isahol{\textsf{Isabelle/HOL}}
\def\isa{\textsf{Isabelle}}
\def\hol{\textsf{HOL}}
\def\ccert{\textsf{CompCert}}
\def\lirmm{\textsf{LIRMM}}
\def\lustre{\textsf{Lustre}}
\def\nasa{\textsf{NASA}}
\def\um{\textsf{Universit\'e de Montpellier}}
\def\neurin{\textsf{NEURINNOV}}
\def\uml{\textsf{UML}}
\def\vhdl{\textsf{VHDL}}
\def\hvhdl{$\mathcal{H}$-\textsf{VHDL}}
\def\lwidth{\dimexpr\linewidth-2\fboxsep-2\fboxrule}
\def\ocaml{\textsf{OCaml}}

%%% Local Variables:
%%% mode: latex
%%% TeX-master: "main"
%%% End:


%%%%

\jyear{2022}%

%% as per the requirement new theorem styles can be included as shown below
\theoremstyle{thmstyleone}%
\newtheorem{theorem}{Theorem}%  meant for continuous numbers
%%\newtheorem{theorem}{Theorem}[section]% meant for sectionwise numbers
%% optional argument [theorem] produces theorem numbering sequence instead of independent numbers for Proposition
\newtheorem{proposition}[theorem]{Proposition}% 
%%\newtheorem{proposition}{Proposition}% to get separate numbers for theorem and proposition etc.

\theoremstyle{thmstyletwo}%
\newtheorem{example}{Example}%
\newtheorem{remark}{Remark}%
\newtheorem*{pf}{Proof}%

\theoremstyle{thmstylethree}%
\newtheorem{definition}{Definition}%
\newtheorem{lemma}{Lemma}%

\raggedbottom

\begin{document}

    \begin{theorem}[Equality between two sums]
      \label{thm:sums-equal}
      For all sets $X,A$ such that $A\subseteq{}X$,
      $f\in{}A\rightarrow\mathbb{N}$, and for all sets $Y, B$, such
      that $Y\subseteq{}X$, and $g\in{}B\rightarrow\mathbb{N}$, if
      there exists $\beta\in{}Y\rightarrow{}B$ such that:
      \begin{itemize}
      \item $\beta$ is a bijection
      \item $\forall{}a\in{}A\cap{}Y,~f(a)=g(\beta(a))$
      \item $\forall{}a\in{}A\setminus{}Y,~f(a)=0$
      \item $\forall{}b\in{}B,\beta^{-1}(b)\notin{}A\Rightarrow{}g(b)=0$
      \end{itemize}
      then $\sum\limits_{a\in{}A}f(a)=\sum\limits_{b\in{}B}g(b)$.
    \end{theorem}

    \begin{pf}[Proof of Theorem~\ref{thm:sums-equal}]

      First, let us define some useful notations to navigate across
      the proof:

      \begin{itemize}
      \item $\mathtt{bij}(\beta)\equiv\beta$ is a bijection
      \item
        $\mathtt{fg}(A,Y,\beta)\equiv\forall{}a\in{}A\cap{}Y,~f(a)=g(\beta(a))$
      \item $\mathtt{f0}(A,Y)\equiv\forall{}a\in{}A\setminus{}Y,~f(a)=0$
      \item
        $\mathtt{g0}(B,A,\beta)\equiv\forall{}b\in{}B,\beta^{-1}(b)\notin{}A\Rightarrow{}g(b)=0$
      \item $\begin{aligned}[t]
               \mathtt{P}(A,\sum\limits_{a\in{}A}f(a))\equiv{}& \begin{aligned}[t]
                                                                  \forall{}Y&\subseteq{}X,B,g\in{}B\rightarrow{}\mathbb{N}, \\
                                                                            & \big(\exists{}\beta\in{}Y\rightarrow{}B~s.t.~\mathtt{bij}(\beta)\land\mathtt{fg}(A,Y,\beta)\land\mathtt{f0}(A,Y)\land\mathtt{g0}(B,A,\beta)\big) \\
                                                                            & \Rightarrow{}\sum\limits_{a\in{}A}f(a)=\sum\limits_{b\in{}B}g(b)\\
                                                                \end{aligned} \\
             \end{aligned}$
      \end{itemize}

      Given a $X, A\subseteq{}X$ and $f\in{}A\rightarrow{}\mathbb{N}$,
      let us show that
      \begin{equation*}
        \boxed{\mathtt{P}(A,\sum\limits_{a\in{}A}f(a)).}
      \end{equation*}

      By induction on $\sum\limits_{a\in{}A}f(a)$, there are two cases
      to prove:
      \begin{enumerate}
      \item $\boxed{\mathtt{P}(\emptyset,0)}$
      \item
        $\boxed{\mathtt{P}(\{a\}\cup{}A',f(a)+\sum_{a'\in{}A'}f(a'))}$
        where $A=\{a\}\cup{}A'$ and such that $A'=A\setminus{}\{a'\}$.
      \end{enumerate}

      \begin{enumerate}
      \item First, let us prove
        \begin{equation*}
          \boxed{\mathtt{P}(\emptyset,0).}
        \end{equation*}

        Given a $Y\subseteq{}X,B,g\in{}B$, and assuming
        $\exists{}\beta\in{}Y\rightarrow{}B~s.t.~\mathtt{bij}(\beta)\land\mathtt{fg}(\emptyset,Y,\beta)\land\mathtt{f0}(\emptyset,Y)\land\mathtt{g0}(B,\emptyset,\beta)$,
        let us show
        \begin{equation*}
          \boxed{0=\sum\limits_{b\in{}B}g(b).}
        \end{equation*}
        Let us take such a $\beta$. From $\mathtt{g0}(B,\emptyset,\beta)$,
        we can deduce that $\forall{}b\in{}B,g(b)=0$, and thus that
        $\sum\limits_{b\in{}B}g(b)=0$.
      \item Then, let us prove that
        \begin{equation*}
          \boxed{\mathtt{P}(\{a\}\cup{}A',f(a)+\sum_{a'\in{}A'}f(a')).}
        \end{equation*}
        The induction hypothesis is
        $\mathtt{P}(A',\sum\limits_{a'\in{}A'}f(a'))$. Given
        $Y\subseteq{}X,B,g\in{}B$, and assuming
        $\exists{}\beta\in{}Y\rightarrow{}B~s.t.~\mathtt{bij}(\beta)\land\mathtt{fg}(\{a\}\cup{}A',Y,\beta)\land\mathtt{f0}(\{a\}\cup{}A',Y)\land\mathtt{g0}(B,\{a\}\cup{}A',\beta)$,
        let us show
        \begin{equation*}
          \boxed{f(a)+\sum_{a'\in{}A'}f(a')=\sum\limits_{b\in{}B}g(b).}
        \end{equation*}
        Let us take such a $\beta$ as described above. Assuming that
        the membership is decidable for the elements of the $Y$ set,
        there are two cases: $a\in{}Y$ or $a\notin{}Y$.

        \begin{enumerate}
        \item \textbf{CASE} $a\in{}Y$, and thus
          $a\in(\{a\}\cup{}A')\cap{}Y$:\\

          Then, from $\mathtt{fg}(\{a\}\cup{}A',Y,\beta)$, we can deduce
          $f(a)=g(\beta(a))$, and the goal can be rewritten as follows:
          \begin{equation*}
            \boxed{f(a)+\sum_{a'\in{}A'}f(a')=g(\beta(a))+\sum\limits_{b\in{}B\setminus\{\beta(a)\}}g(b).}
          \end{equation*}
          From $f(a)=g(\beta(a))$, it is now sufficient to prove that
          \begin{equation*}
            \boxed{\sum_{a'\in{}A'}f(a')=\sum\limits_{b\in{}B\setminus\{\beta(a)\}}g(b).}
          \end{equation*}

          Appealing to the induction hypothesis with
          $B:=B\setminus\{\beta(a)\}$, $Y:=Y\setminus\{a\}$ and
          $g:=g$, it is now sufficient two show that
          \begin{equation*}
            \boxed{
              \begin{aligned}[t]
                \exists{}\beta\in{}&Y\setminus\{a\}\rightarrow{}B\setminus\{\beta(a)\}~s.t.~\\
                                   & \mathtt{bij}(\beta)\land\mathtt{fg}(A',Y\setminus\{a\},\beta)\land\mathtt{f0}(A',Y\setminus\{a\})\land\mathtt{g0}(B\setminus\{\beta(a)\},A',\beta) \\
              \end{aligned}
            }
          \end{equation*}

          Let us take $\beta$ restricted to the domain of
          $Y\setminus\{a\}$, and called it $\beta_r$, to prove the
          goal. By definition of the restriction,
          $\beta_r\in{}Y\setminus\{a\}\rightarrow{}B\setminus\{\beta(a)\}$. Then,
          we must show the four following points:
          \begin{enumerate}
          \item $\mathtt{bij}(\beta_r)$
          \item $\mathtt{fg}(A',Y\setminus\{a\},\beta_r)$
          \item $\mathtt{f0}(A',Y\setminus\{a\})$
          \item $\mathtt{g0}(B_r\setminus\{\beta(a)\},A',\beta_r)$
          \end{enumerate}

          \vspace{10pt}
          
          \begin{enumerate}
          \item $\boxed{\mathtt{bij}(\beta_r).}$ Trivial by definition of $\beta_r$.
          \item $\boxed{\mathtt{fg}(A',Y\setminus\{a\},\beta_r).}$
            Then, we must show
            \begin{equation*}
              \boxed{\forall{}a'\in{}A'\cap(Y\setminus\{a\}),f(a')=g(\beta_r(a')).}
            \end{equation*}
            Given a $a'\in{}A'\cap(Y\setminus\{a\})$, let us show
            \begin{equation*}
              \boxed{f(a')=g(\beta_r(a')).}
            \end{equation*}
            As $a'\in{}A'\cap(Y\setminus\{a\})$, then
            $a'\in{}(\{a\}\cup{}A')\cap{}Y$ and $a'\neq{}a$. Thanks to
            the previous assumption that
            $\mathtt{fg}(\{a\}\cup{}A',Y,\beta)$,
            $a'\in{}(\{a\}\cup{}A')\cap{}Y$ and $a'\neq{}a$, we can
            deduce $f(a')=g(\beta(a'))$ and $\beta(a')=\beta_r(a')$
            (since $a'\neq{}a$). Thus, we have $f(a')=g(\beta_r(a'))$.
            
          \item $\boxed{\mathtt{f0}(A',Y\setminus\{a\}).}$ Then, we must show
            \begin{equation*}
              \boxed{\forall{}a'\in{}A'\setminus(Y\setminus\{a\}),f(a')=0.}
            \end{equation*}

            Given a $a'\in{}A'\setminus(Y\setminus\{a\})$, let us show
            \begin{equation*}
              \boxed{f(a')=0.}
            \end{equation*}

            By definition, $a'\in{}A'$ and
            $a'\notin{}Y\setminus\{a\}$. Then, from the previous
            assumption that $\mathtt{f0}(\{a\}\cup{}A',Y)$, it is
            sufficient to show that
            \begin{equation*}
              \boxed{a'\in(\{a\}\cup{}A')\setminus{}Y.}
            \end{equation*}

            Then, it is sufficient to show the two following points:
            \begin{enumerate}
            \item $\boxed{a'\in\{a\}\cup{}A'.}$
            \item $\boxed{a'\notin{}Y.}$
            \end{enumerate}

            \vspace{10pt}

            \begin{enumerate}
            \item $\boxed{a'\in\{a\}\cup{}A'.}$ Trivial as $a'\in{}A'$.
            \item $\boxed{a'\notin{}Y.}$ Since
              $a'\notin{}Y\setminus\{a\}$, then either $a'\notin{}Y$,
              or $a'=a$ which contradicts $a'\in{}A'$ as we know that
              $a\notin{}A'$.
            \end{enumerate}
            
          \item $\boxed{\mathtt{g0}(B\setminus\{\beta_r(a)\},A',\beta_r).}$ Then, we must show
            \begin{equation*}
              \boxed{\forall{}b\in{}B\setminus\{\beta_r(a)\},\beta_r^{-1}(b)\notin{}A'\Rightarrow{}g(b)=0.}
            \end{equation*}

            Given $b\in{}B\setminus\{\beta_r(a)\}$ and assuming that
            $\beta_r^{-1}(b)\notin{}A'$, let us show
            \begin{equation*}
              \boxed{g(b)=0.}
            \end{equation*}

            From the previous assumption
            $\mathtt{g0}(B,\{a\}\cup{}A',\beta)$, it is sufficient to
            show the two following points:
            \begin{enumerate}
            \item $\boxed{b\in{}B.}$
            \item $\boxed{\beta^{-1}(b)\notin\{a\}\cup{}A'.}$ 
            \end{enumerate}

            \vspace{10pt}

            \begin{enumerate}
            \item $\boxed{b\in{}B.}$ Trivial from
              $b\in{}B\setminus{\beta_r(a)}$.
            \item $\boxed{\beta^{-1}(b)\notin\{a\}\cup{}A'.}$ Then, we
              must show the two following points:

              \begin{itemize}
              \item $\boxed{\beta^{-1}(b)\neq{}a.}$ If
                $\beta^{-1}(b)=a$, then $\beta_r^{-1}(b)\notin{}A'$ is
                a contradiction since $b$ is not defined in the domain
                of $\beta_r^{-1}$. Otherwise, $\beta^{-1}(b)\neq{}a$.
              \item $\boxed{\beta^{-1}(b)\notin{}A'.}$ Trivial from
                $\beta_r^{-1}(b)\notin{}A'$ and knowing now that
                $\beta^{-1}(b)\neq{}a$.
              \end{itemize}
            \end{enumerate}
            
          \end{enumerate}

          \vspace{10pt}
          
        \item \textbf{CASE} $a\notin{}Y$ and thus $a\in{}(\{a\}\cup{}A')\setminus{}Y$:

          Then, from $\mathtt{f0}(\{a\}\cup{}A',Y)$, we can deduce
          $f(a)=0$ and the goal can be rewritten as follows:
          \begin{equation*}
            \boxed{\sum\limits_{a'\in{}A'}f(a')=\sum\limits_{b\in{}B}g(b).}
          \end{equation*}
          
        \end{enumerate}
      \end{enumerate}
      
    \end{pf}

\end{document}


%%% Local Variables:
%%% mode: latex
%%% TeX-master: t
%%% End:
