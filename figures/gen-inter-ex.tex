\documentclass{standalone}

\usepackage[english]{babel}

% to define font size

\usepackage{ulem}
\usepackage{moresize}
\usepackage{anyfontsize}

% to use colors

\usepackage[dvipsnames]{xcolor}
\usepackage{MnSymbol}

% to use tikz and its libraries

\usepackage{tikz-timing}
\usepackage{tikz}

\usetikzlibrary{backgrounds}
\usetikzlibrary{positioning, calc, arrows, shapes, automata, petri, patterns,decorations.markings}

% to use tikzmark, to place and refer to marks outside the current figure

\tikzset{every picture/.style={remember picture}}

% styles for transitions

\tikzset{transition/.append style={fill=black!20, thick}}
\tikzset{transition/.append style={fill=black!20, thick}}

% styles for test and inhib arcs.

\tikzstyle{test}=[pre, *-]
\tikzstyle{inhib}=[pre, o-]

\usepackage{circuitikzgit}
\ctikzset{
  logic ports=ieee,
}

% Arrow positioning in a path

\tikzset{->-/.style={decoration={
  markings,
  mark=at position #1 with {\arrow{>}}},postaction={decorate}}}

\tikzset{-<-/.style={decoration={
  markings,
  mark=at position #1 with {\arrow{<}}},postaction={decorate}}}

% shift values

\newcommand{\outportshift}{0mm}
\newcommand{\outportidpshift}{0mm}

%%%%%%%%%%%%%%%%%%%%%%%%%%%%%%%%%%%%%%%%%%%%%%%%%%
%                  BEGIN DOCUMENT                %
%%%%%%%%%%%%%%%%%%%%%%%%%%%%%%%%%%%%%%%%%%%%%%%%%%

\begin{document}

\begin{circuitikz}

  \ctikzset{multipoles/dipchip/width=2}
  \ctikzset{multipoles/dipchip/pin spacing=.18}

  % PCI idp

  % interface
  \draw       
  node [dipchip, num pins=16, hide numbers,
  no topmark, external pins width=0]
  (idp) {$\mathtt{id}_{p_0}$};

  \draw ($(idp.bpin 1)$) ++(0,0.1) -- ++(0.1,-0.1) node[right, font=\ssmall] {\tt clk} -- ++(-0.1,-0.1) ;
  
  \draw ($(idp.bpin 2)$) --++ (-.3,0) coordinate (idprst);
  \node at (idp.bpin 2) [anchor=west, font=\ssmall]  {\tt rst};

  \draw ($(idp.bpin 3)$) -- ++(-.3,0) coordinate (idpim);
  \node at (idp.bpin 3) [anchor=west, font=\ssmall]  {\tt im};
  \node at (idpim) [anchor=east, font=\ssmall,xshift=\outportshift] {\tt \textcolor{blue}{1}};
  
  \draw ($(idp.bpin 4)$) -- ++(-.3,0) coordinate (idpiaw0);
  \node at (idp.bpin 4) [anchor=west, font=\ssmall]  {\tt iaw(\textcolor{blue}{0})};
  \node at (idpiaw0) [anchor=east, font=\ssmall,xshift=\outportshift] {\tt \textcolor{blue}{1}};

  \draw ($(idp.bpin 5)$) -- ++(-.3,0) coordinate (idpoat0);
  \node at (idp.bpin 5) [anchor=west, font=\ssmall]  {\tt oat(\textcolor{blue}{0})};
  \node at (idpoat0) [anchor=east, font=\ssmall,xshift=\outportshift] {\tt \textcolor{blue}{basic}};

  \draw ($(idp.bpin 6)$) -- ++(-.3,0) coordinate (idpoaw0);
  \node at (idp.bpin 6) [anchor=west, font=\ssmall]  {\tt oaw(\textcolor{blue}{0})};
  \node at (idpoaw0) [anchor=east, font=\ssmall,xshift=\outportshift] {\tt \textcolor{blue}{1}};
  
  \draw[red,-<-=.4] ($(idp.bpin 7)$) --++ (-.4,0) coordinate (idpitf0) node[anchor=east] {\ssmall\tt \dots};
  \node at (idp.bpin 7) [anchor=west, font=\ssmall]  {\tt itf(\textcolor{blue}{0})};

  \coordinate (idpotf0) at ($(idp.bpin 8)+(-.3,0)$);
  \node at (idp.bpin 8) [anchor=west, font=\ssmall]  {\tt otf(\textcolor{blue}{0})};

  \coordinate (idpoav0) at ($(idp.bpin 15)+(.3,0)$);
  \node at (idp.bpin 15) [anchor=east, font=\ssmall, xshift=\outportidpshift]  {\tt oav(\textcolor{blue}{0})};
  
  \coordinate (idprtt0) at ($(idp.bpin 13)+(.3,0)$);
  \node at (idp.bpin 13) [anchor=east, font=\ssmall, xshift=\outportidpshift]  {\tt rtt(\textcolor{blue}{0})};

  \draw ($(idp.bpin 11)$) -- ++(.3,0) coordinate (idppauths0) ;

  \node at (idp.bpin 11) [anchor=east, font=\ssmall, xshift=\outportidpshift]  {\tt pauths(\textcolor{blue}{0})};

  \draw[red,->-=.4] ($(idp.bpin 9)$) --++(.6,0) coordinate (idpmarked);
  \node at (idp.bpin 9) [anchor=east, font=\ssmall, xshift=\outportidpshift]  {\tt m};
  
  %   TCI idt
  \ctikzset{multipoles/dipchip/width=2.2}
  \draw       
  node [dipchip, anchor=west, num pins=16, hide numbers,
  no topmark, external pins width=0]
  (idt) at ($(idp)+(4,0)$) {$\mathtt{id}_{t_0}$};

  \draw ($(idt.bpin 1)$) ++(0,0.1) -- ++(0.1,-0.1) node[right, font=\ssmall] {\tt clk} -- ++(-0.1,-0.1) ;
  
  \draw ($(idt.bpin 2)$) --++ (-.3,0) coordinate (idtrst);
  \node at (idt.bpin 2) [anchor=west, font=\ssmall]  {\tt rst};

  \draw ($(idt.bpin 4)$) -- ($(idt.bpin 4)+(-.3,0)$) coordinate (idtA);
  \node at (idt.bpin 4) [anchor=west, font=\ssmall]  {\tt A};
  \node at (idtA) [anchor=east, font=\ssmall,xshift=\outportshift] {\tt \textcolor{blue}{1}};
  
  \draw ($(idt.bpin 5)$) -- ($(idt.bpin 5)+(-.3,0)$) coordinate (idtB);
  \node at (idt.bpin 5) [anchor=west, font=\ssmall]  {\tt B};
  \node at (idtB) [anchor=east, font=\ssmall,xshift=\outportshift] {\tt \textcolor{blue}{3}};

  \draw ($(idt.bpin 6)$) -- ++(-.3,0) coordinate (idtpauths0);
  \node at (idtpauths0) [anchor=east, font=\ssmall] {\tt\ssmall true};
  \node at (idt.bpin 6) [anchor=west, font=\ssmall]  {\tt pauths(\textcolor{blue}{0})};
  
  \coordinate (idtiav0) at ($(idt.bpin 7)+(-.3,0)$);
  \node at (idt.bpin 7) [anchor=west, font=\ssmall]  {\tt iav(\textcolor{blue}{0})};

  \coordinate (idtrt0) at ($(idt.bpin 8)+(-.3,0)$);
  \node at (idt.bpin 8) [anchor=west, font=\ssmall]  {\tt rt(\textcolor{blue}{0})};

  \draw ($(idt.bpin 3)$) --++ (-.3,0) coordinate (idtic0);
  \node at (idt.bpin 3) [anchor=west, font=\ssmall]  {\tt ic(\textcolor{blue}{0})};

  \draw[red,->-=.4] ($(idt.east)$) --++ (.6,0) coordinate (idtfired);
  \node at (idt.east) [anchor=east, font=\ssmall, xshift=\outportshift]  {\tt f};

  % INTERCONNECTIONS in INTERNAL BEHAVIOR
  
  % idt.fired to idp.itf/idp.otf/functions
  \draw [red,->-=.4] (idt.east) -- (idtfired) |- ($(idpitf0)-(0,.8)$) |- (idp.bpin 8);
  % \draw [Orange] (idt.east) -- (idtfired) |- ($(idpitf0)-(0,.7)$) |- (idp.bpin 7);
  % \draw [Orange,->-=.4] (idt.east) -- (idtfired) |- (funpsf) --++(.3,0);
  
  % idp.oav(0) to idt.iav(0)
  \draw [red,->-=.4] (idp.bpin 15) -- (idpoav0) -| ($(idtiav0)-(.7,0)$) -- (idt.bpin 7);

  % idp.rtt(0) to idt.rt(0)
  \draw [red,->-=.3] (idp.bpin 13) -- (idprtt0) -| ($(idtrt0)-(.9,0)$) -- (idt.bpin 8);

  % idp.marked to actions
  % \draw [Orange,->-=.4] (idp.bpin 15) -- (idpmarked) -| ($(actpsm)-(.6,0)$) -- (actps.bpin 3);
  
  % \draw [->-=.4] (tlclk) --++(1,0) -|++ (0,-2.6) -- (idp.bpin 1);
  % \draw [->-=.4] (tlclk) --++(1,0) -|++ (0,2) -| ($(funps.bpin 1)-(.3,0)$) -- ($(funps.bpin 1)$);
  % \draw (tlclk) --++(1,0) -|++ (0,2) -| ($(actps.bpin 1)-(.8,0)$) -- ($(actps.bpin 1)$);
  
\end{circuitikz}


\end{document}
%%% Local Variables:
%%% mode: latex
%%% TeX-master: t
%%% End:
