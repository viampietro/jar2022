\documentclass{standalone}

\usepackage[english]{babel}

% to define font size

\usepackage{ulem}
\usepackage{moresize}
\usepackage{anyfontsize}

% to use colors

\usepackage[dvipsnames]{xcolor}
\usepackage{MnSymbol}

% to use tikz and its libraries

\usepackage{tikz-timing}
\usepackage{tikz}

\usetikzlibrary{backgrounds}
\usetikzlibrary{positioning, calc, arrows, shapes, automata, petri, patterns,decorations.markings}

% to use tikzmark, to place and refer to marks outside the current figure

\tikzset{every picture/.style={remember picture}}

% styles for transitions

\tikzset{transition/.append style={fill=black!20, thick}}
\tikzset{transition/.append style={fill=black!20, thick}}

% styles for test and inhib arcs.

\tikzstyle{test}=[pre, *-]
\tikzstyle{inhib}=[pre, o-]

\usepackage{circuitikzgit}
\ctikzset{
  logic ports=ieee,
}

% Arrow positioning in a path

\tikzset{->-/.style={decoration={
  markings,
  mark=at position #1 with {\arrow{>}}},postaction={decorate}}}

\tikzset{-<-/.style={decoration={
  markings,
  mark=at position #1 with {\arrow{<}}},postaction={decorate}}}

% shift values

\newcommand{\outportshift}{6mm}
\newcommand{\outportidpshift}{6mm}

%%%%%%%%%%%%%%%%%%%%%%%%%%%%%%%%%%%%%%%%%%%%%%%%%%
%                  BEGIN DOCUMENT                %
%%%%%%%%%%%%%%%%%%%%%%%%%%%%%%%%%%%%%%%%%%%%%%%%%%

\begin{document}

\begin{circuitikz}

  \ctikzset{multipoles/dipchip/width=2}
  \ctikzset{multipoles/dipchip/pin spacing=.18}

  \node[draw,rectangle,inner sep=7mm] (tl) {
    \begin{circuitikz}

      % PCI idp

      % interface
      \draw       
      node [dipchip, num pins=16, hide numbers,
      no topmark, external pins width=0]
      (idp) {$\mathtt{id}_p$};

      \draw ($(idp.bpin 1)$) ++(0,0.1) -- ++(0.1,-0.1) node[right, font=\ssmall,xshift=-6mm] {\tt clk} -- ++(-0.1,-0.1) ;
      
      \coordinate (idprst) at ($(idp.bpin 2)+(-.3,0)$);
      \node at (idp.bpin 2) [anchor=west, font=\ssmall, xshift=-6mm]  {\tt rst};

      \draw ($(idp.bpin 3)$) -- ++(-.3,0) coordinate (idpim);
      \node at (idp.bpin 3) [anchor=west, font=\ssmall, xshift=-6mm]  {\tt im};
      \node at (idpim) [anchor=east, font=\ssmall,xshift=\outportshift] {\tt 1};
      
      \draw ($(idp.bpin 4)$) -- ++(-.3,0) coordinate (idpiaw0);
      \node at (idp.bpin 4) [anchor=west, font=\ssmall, xshift=-6mm]  {\tt iaw(0)};
      \node at (idpiaw0) [anchor=east, font=\ssmall,xshift=\outportshift] {\tt 1};

      \draw ($(idp.bpin 5)$) -- ++(-.3,0) coordinate (idpoat0);
      \node at (idp.bpin 5) [anchor=west, font=\ssmall, xshift=-6mm]  {\tt oat(0)};
      \node at (idpoat0) [anchor=east, font=\ssmall,xshift=\outportshift] {\tt basic};

      \draw ($(idp.bpin 6)$) -- ++(-.3,0) coordinate (idpoaw0);
      \node at (idp.bpin 6) [anchor=west, font=\ssmall, xshift=-6mm]  {\tt oaw(0)};
      \node at (idpoaw0) [anchor=east, font=\ssmall,xshift=\outportshift] {\tt 1};
      
      \coordinate (idpitf0) at ($(idp.bpin 7)+(-.3,0)$);
      \node at (idp.bpin 7) [anchor=west, font=\ssmall, xshift=-6mm]  {\tt itf(0)};

      \coordinate (idpotf0) at ($(idp.bpin 8)+(-.3,0)$);
      \node at (idp.bpin 8) [anchor=west, font=\ssmall, xshift=-6mm]  {\tt otf(0)};

      \coordinate (idpmarked) at ($(idp.bpin 15)+(.3,0)$);
      \node at (idp.bpin 15) [anchor=east, font=\ssmall, xshift=\outportidpshift]  {\tt marked};

      \coordinate (idprtt0) at ($(idp.bpin 13)+(.3,0)$);
      \node at (idp.bpin 13) [anchor=east, font=\ssmall, xshift=\outportidpshift]  {\tt rtt(0)};

      \draw ($(idp.bpin 11)$) -- ++(.3,0) coordinate (idppauths0);
      \node at (idp.bpin 11) [anchor=east, font=\ssmall, xshift=\outportidpshift]  {\tt pauths(0)};
      
      \coordinate (idpoav0) at ($(idp.bpin 9)+(.3,0)$);
      \node at (idp.bpin 9) [anchor=east, font=\ssmall, xshift=\outportidpshift]  {\tt oav(0)};
      
      % TCI idt
      \ctikzset{multipoles/dipchip/width=2.2}
      \draw       
      node [dipchip, anchor=east, num pins=16, hide numbers,
      no topmark, external pins width=0]
      (idt) at ($(idp.east)+(5,3)$) {$\mathtt{id}_t$};

      \draw ($(idt.bpin 1)$) ++(0,0.1) -- ++(0.1,-0.1) node[right, font=\ssmall,xshift=-6mm] {\tt clk} -- ++(-0.1,-0.1) ;
      
      \coordinate (idtrst) at ($(idt.bpin 2)+(-.3,0)$);
      \node at (idt.bpin 2) [anchor=west, font=\ssmall, xshift=-6mm]  {\tt rst};

      \draw ($(idt.bpin 3)$) -- ($(idt.bpin 3)+(-.3,0)$) coordinate (idtA);
      \node at (idt.bpin 3) [anchor=west, font=\ssmall, xshift=-6mm]  {\tt A};
      \node at (idtA) [anchor=east, font=\ssmall,xshift=\outportshift] {\tt 0};
      
      \draw ($(idt.bpin 4)$) -- ($(idt.bpin 4)+(-.3,0)$) coordinate (idtB);
      \node at (idt.bpin 4) [anchor=west, font=\ssmall, xshift=-6mm]  {\tt B};
      \node at (idtB) [anchor=east, font=\ssmall,xshift=\outportshift] {\tt 0};

      \draw ($(idt.bpin 5)$) -- ++(-.3,0) coordinate (idtpauths0);
      \node at (idt.bpin 5) [anchor=west, font=\ssmall, xshift=-6mm]  {\tt pauths(0)};
      \node at (idtpauths0) [anchor=east, font=\ssmall,xshift=\outportshift] {\tt true};
      
      \coordinate (idtiav0) at ($(idt.bpin 6)+(-.3,0)$);
      \node at (idt.bpin 6) [anchor=west, font=\ssmall, xshift=-6mm]  {\tt iav(0)};

      \coordinate (idtrt0) at ($(idt.bpin 7)+(-.3,0)$);
      \node at (idt.bpin 7) [anchor=west, font=\ssmall, xshift=-6mm]  {\tt rt(0)};

      \coordinate (idtic0) at ($(idt.bpin 8)+(-.3,0)$);
      \node at (idt.bpin 8) [anchor=west, font=\ssmall, xshift=-6mm]  {\tt ic(0)};

      \coordinate (idtfired) at ($(idt.east)+(.3,0)$);
      \node at (idt.east) [anchor=east, font=\ssmall, xshift=\outportshift]  {\tt fired};
      
      % Process functions
      \ctikzset{multipoles/dipchip/width=2.2}
      \draw       
      node [dipchip, anchor=east, num pins=6, hide numbers,
      no topmark, external pins width=0]
      (funps) at ($(idt.east)+(5,0)$) {$\mathtt{functions}$};

      \draw ($(funps.bpin 1)$) ++(0,0.1) -- ++(0.1,-0.1) node[right, font=\ssmall,xshift=-6mm] {\tt clk} -- ++(-0.1,-0.1) ;
      
      \coordinate (funpsrst) at ($(funps.bpin 2)+(-.3,0)$);
      \node at (funps.bpin 2) [anchor=west, font=\ssmall, xshift=-6mm]  {\tt rst};
      
      \coordinate (funpsf) at ($(funps.bpin 3)+(-.3,0)$);
      \node at (funps.bpin 3) [anchor=west, font=\ssmall, xshift=-6mm]  {};
      
      % Process actions
      \draw       
      node [dipchip, anchor=north, num pins=6, hide numbers,
      no topmark, external pins width=0]
      (actps) at ($(funps.south)-(0,2)$) {$\mathtt{actions}$};

      \draw ($(actps.bpin 1)$) ++(0,0.1) -- ++(0.1,-0.1) node[right, font=\ssmall,xshift=-6mm] {\tt clk} -- ++(-0.1,-0.1) ;
      
      \coordinate (actpsrst) at ($(actps.bpin 2)+(-.3,0)$);
      \node at (actps.bpin 2) [anchor=west, font=\ssmall, xshift=-6mm]  {\tt rst};
      
      \coordinate (actpsm) at ($(actps.bpin 3)+(-.3,0)$);
      \node at (actps.bpin 3) [anchor=west, font=\ssmall, xshift=-6mm] {};
    \end{circuitikz}
  };

  % TOP-LEVEL PORTS
  
  \draw ($(tl.west)+(0,2)$) ++(0,0.1) -- ++(0.1,-0.1) coordinate (tlclk) node[left, xshift=-4mm]{\tt clk} -- ++(-0.1,-0.1);
  \draw ($(tl.west)+(0,1)$) -- ++(-.3,0) coordinate (tlrst) node [left] {\tt rst};
  \draw ($(tl.west)$) -- ++(-.3,0) coordinate (idc0) node [left] {\tt id$_{c_0}$};

  \draw ($(tl.east)+(0,1)$) -- ++(.3,0) coordinate (idf0) node [right] {\tt id$_{f_0}$};

  \draw ($(tl.east)+(0,-1)$) -- ++(.3,0) coordinate (ida0) node [right] {\tt id$_{a_0}$};

  % INTERCONNECTIONS between TL and INTERNAL BEHAVIOR

  % clk to clks
  \draw [red,->-=.4] (tlclk) --++(1,0) |- (idt.bpin 1);
  \draw [red,->-=.4] (tlclk) --++(1,0) -|++ (0,-2.6) -- (idp.bpin 1);
  \draw [red,->-=.4] (tlclk) --++(1,0) -|++ (0,1.5) -| ($(funps.bpin 1)-(.3,0)$) -- ($(funps.bpin 1)$);
  \draw [red](tlclk) --++(1,0) -|++ (0,1.5) -| ($(actps.bpin 1)-(.8,0)$) -- ($(actps.bpin 1)$);

  % rst to rsts
  \draw [blue,->-=.4] (tlrst) --++(3,0) |- (idt.bpin 2);
  \draw [blue,->-=.4] (tlrst) --++(.7,0) |- (idp.bpin 2);
  \draw [blue,->-=.4] (tlrst) --++(.7,0) --++ (0,2.7) -| ($(funps.bpin 2)-(.5,0)$) -- ($(funps.bpin 2)$);
  \draw [blue] (tlrst) --++(.7,0) --++ (0,2.7) -| ($(funps.bpin 2)-(.5,0)$) |- ($(actps.bpin 2)$);

  % idc0 to ic(0)
  \draw [Green,->-=.4] (idc0) -|  (idtic0) -- ($(idtic0)+(.3,0)$);

  % functions to idf0
  \draw [Purple,->-=.2] (funps.east) -| ($(idf0)-(.6,0)$) -- (idf0);

  % actions to ida0
  \draw [Purple,->-=.2] (actps.east) -| ($(ida0)-(.6,0)$) -- (ida0);

  % INTERCONNECTIONS in INTERNAL BEHAVIOR
  
  % idt.fired to idp.itf/idp.otf/functions
  \draw [Orange,->-=.4] (idt.east) -- (idtfired) |- ($(idpitf0)-(0,.7)$) |- (idp.bpin 8);
  \draw [Orange] (idt.east) -- (idtfired) |- ($(idpitf0)-(0,.7)$) |- (idp.bpin 7);
  \draw [Orange,->-=.4] (idt.east) -- (idtfired) |- (funpsf) --++(.3,0);
  
  % idp.oav(0) to idt.iav(0)
  \draw [Orange,->-=.4] (idp.bpin 9) -- (idpoav0) -| ($(idtiav0)-(.2,0)$) -- (idt.bpin 6);

  % idp.rtt(0) to idt.rt(0)
  \draw [Orange,->-=.4] (idp.bpin 13) -- (idprtt0) -| ($(idtrt0)-(.4,0)$) -- (idt.bpin 7);

  % idp.marked to actions
  \draw [Orange,->-=.4] (idp.bpin 15) -- (idpmarked) -| ($(actpsm)-(.6,0)$) -- (actps.bpin 3);
  
  % \draw [->-=.4] (tlclk) --++(1,0) -|++ (0,-2.6) -- (idp.bpin 1);
  % \draw [->-=.4] (tlclk) --++(1,0) -|++ (0,2) -| ($(funps.bpin 1)-(.3,0)$) -- ($(funps.bpin 1)$);
  % \draw (tlclk) --++(1,0) -|++ (0,2) -| ($(actps.bpin 1)-(.8,0)$) -- ($(actps.bpin 1)$);
  
\end{circuitikz}


\end{document}
%%% Local Variables:
%%% mode: latex
%%% TeX-master: t
%%% End:
