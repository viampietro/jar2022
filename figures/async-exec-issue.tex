\documentclass{standalone}

\usepackage[english]{babel}
\usepackage[linesnumbered, ruled, vlined]{algorithm2e}

\usepackage{caption}

% to inlcude graphics

\usepackage{graphicx}
\usepackage{subfig}

% to create listings

\usepackage{listings, lstautogobble}
\lstset{
  autogobble=true,
  frame=single,
}

\lstdefinelanguage{coq}[Objective]{Caml}{
  morekeywords={Structure, Definition, Inductive, list, return},
  sensitive=true
}

% to define font size

\usepackage{ulem}
\usepackage{moresize}
\usepackage{anyfontsize}

% to use tikz and its libraries

\usepackage{tikz-timing}
\usepackage{tikz}

\usetikzlibrary{backgrounds}
\usetikzlibrary{positioning, calc, arrows, shapes, automata, petri, patterns}

% to use tikzmark, to place and refer to marks outside the current figure

\tikzset{every picture/.style={remember picture}}

% styles for transitions

\tikzset{transition/.append style={fill=black!20, thick}}
\tikzset{transition/.append style={fill=black!20, thick}}

% styles for test and inhib arcs.

\tikzstyle{test}=[pre, *-]
\tikzstyle{inhib}=[pre, o-]

% to use colors

\usepackage{xcolor}

% to frame text

\usepackage{framed}


%%%%%%%%%%%%%%%%%%%%%%%%%%%%%%%%%%%%%%%%%%%%%%%%%%
%                  BEGIN DOCUMENT                %
%%%%%%%%%%%%%%%%%%%%%%%%%%%%%%%%%%%%%%%%%%%%%%%%%%

\begin{document}

% \begin{tikzpicture}
%   \node[place, tokens=1] (p0) [label={left:\ssmall $P_0$}] {};
%   \node[place, tokens=1] (p1) at ($(p0)-(0, 2)$) [label={left:\ssmall $P_1$}] {};

%   \node[transition] (t0) at ($(p0)!0.5!(p1)$) [label={left:\ssmall $T_0$}] {}
%   edge[pre] (p0)
%   edge[post] (p1);
%   \node (f0) at ($(t0.east)$) [xshift=5pt] {\ssmall $f_0$};
  
%   \node[transition] (t1) at ($(p1)-(0, 1)$) [label={left:\ssmall $T_1$}] {}
%   edge[pre] (p1);
%   \node (c0) at ($(t1.east)$) [xshift=5pt] {\ssmall $c_0$};
%   \draw ($(t1.south west)-(.5, .5)$) -- ($(t1.south east)+(.5, -.5)$);
%   \node [below =6mm of t1, align=left] {
%     \ssmall $f_0$: $x := 0; y := 1;$ \\
%     \ssmall $c_0$: $x\ \&\&\ y;$
%   };
  
% \end{tikzpicture}
% \begin{tikztimingtable}
%   \ssmall Firing($T_0$) & 2LG9L \\
%   \ssmall Firing($T_1$) & 8LG3L \\
%   \ssmall Launching($f_0$) & 3LG N(startf0) 8L \\
%   \ssmall $x$ & 9H N(endf0) 2L \\
%   \ssmall $y$ & 4LN(a2)7H \\
%   \ssmall $c_0$ & 5L5H1L \\
%   \extracode
%   \draw[dotted] (startf0) -- ($(startf0)+(0,7)$);
%   \draw[dotted] (endf0) -- ($(endf0)+(0,8)$);
%   \draw[<->] ($(startf0)+(0,6.5)$) -- ($(endf0)+(0,7.5)$);
%   \node (s0) at ($(startf0)+(0,6.5)$) {};
%   \node (e0) at ($(endf0)+(0,7.5)$) {};
  
%   \node at ($(s0)!0.5!(e0)+(0,1)$) {\ssmall $f_0$ time duration};
% \end{tikztimingtable}

% \begin{tikztimingtable}
%   \ssmall Firing($T_0$) & 2LG9L \\
%   \ssmall Firing($T_1$) & 11L \\
%   \ssmall Launching($f_0$) & 3LG N(startf0) 8L \\
%   \ssmall $x$ & 9H N(endf0) 2L \\
%   \ssmall $y$ & 4LN(a2)7H \\
%   \ssmall $c_0$ & 5L5H1L \\
%   \extracode
%   \draw[dotted] (startf0) -- ($(startf0)+(0,7)$);
%   \draw[dotted] (endf0) -- ($(endf0)+(0,8)$);
%   \draw[<->] ($(startf0)+(0,6.5)$) -- ($(endf0)+(0,7.5)$);
%   \node (s0) at ($(startf0)+(0,6.5)$) {};
%   \node (e0) at ($(endf0)+(0,7.5)$) {};
  
%   \node at ($(s0)!0.5!(e0)+(0,1)$) {\ssmall $f_0$ time duration};
% \end{tikztimingtable}

\begin{figure}[h]
  \centering
  \hspace{10pt}
  \subfloat[\ssmall An example of Petri net.]{
    \includegraphics[keepaspectratio=true]{async-fig1.eps}
  }\hspace{10pt}
  \subfloat[\ssmall Case: $T_1$ is fired before the end of $f_0$.]{
    \includegraphics[keepaspectratio=true]{async-fig2.eps}
  }\hspace{10pt}
  \subfloat[\ssmall Case: $f_0$ terminates before taking the firing decision for $T_1$.]{
    \includegraphics[keepaspectratio=true]{async-fig3.eps}
  }\hspace{10pt}
\end{figure}


\end{document}
%%% Local Variables:
%%% mode: latex
%%% TeX-master: t
%%% End:
